\section{Ingegneria dell'usabilità}

È la disciplina che studia le tecniche, i metodi, e i processi usati per progettare sistemi usabilili; è seguita poi dall'effetiva realizzazione del prodotto, e dalla verifica delle sue caratteristiche.

Dal disegno della semplice interfaccia utente, si è espansa per comprendere le pratiche usate nel processo di progettazione di sistemi interattivi.

\subsection{Modello iterativo}

\begin{tikzpicture}[
    % Definizione degli stili per mantenere il codice pulito
    state/.style={ellipse, draw=black, minimum width=3.5cm, minimum height=1.5cm, font=\sffamily, fill=white},
    arrow/.style={red, -{Triangle[length=5mm, width=6mm]}, line width=2mm},
    note/.style={fill=yellow, font=\sffamily\footnotesize, align=center, inner sep=3pt}
    ]

    % 1. Posizionamento dei nodi principali (le ellissi)
    \node[state] (req) at (0, 2.5) {Requisiti};
    \node[state] (prog) at (-3.5, 0) {Progettazione};
    \node[state] (prot) at (0, -2.5) {Prototipazione};
    \node[state] (test) at (3.5, 0) {Test};

    % 2. Frecce interne del ciclo (leggermente curvate con 'bend right')
    \draw[arrow] (req) to[bend right=25] (prog);
    \draw[arrow] (prog) to[bend right=25] (prot);
    \draw[arrow] (prot) to[bend right=25] (test);
    \draw[arrow] (test) to[bend right=25] (req);

    % 3. Freccia in entrata (Inizio)
    \draw[arrow] (-5.5, 2.5) -- node[above, font=\sffamily, text=black] {Inizio} (req.west);

    % 4. Freccia in uscita (Rilascio)
    \draw[arrow] (test.east) -- node[above, font=\sffamily, text=black] {Rilascio} (6.5, 0);

    % 5. Annotazioni evidenziate in giallo
    \node[note] at (2.5, 3.8) {Nota bene: anche i requisiti evolvono durante il processo};
    \node[note] at (-3.5, -3.2) {Prototipi perfezionati e sempre pi\`u complessi};
    \node[note] at (6, -2.2) {Le prove d'uso diventano parte\\integrante del processo di progettazione};

\end{tikzpicture}


Si realizza una serie di prototipi sempre più vicini al sistema finale, sottoposti all'utente e aggiustati via via fino alla conclusione del progetto.

Nelle varie iterazioni, man mano ci si sposta dalla fase di raccolta requisiti e progettazione alla fase di test e rilascio del prodotto finito.

\begin{tabularx}{\textwidth}{|c|>{\centering\arraybackslash}X|}
    \hline
    \bld{Pro}                                         & \bld{Contro}                                          \\
    \hline
    Modello concettualmente corretto                  & Difficoltà di stima di costi                          \\
    Prodotto visto fin dall'inizio e perfezionato     & Rischio di esplosione del numero di richieste         \\
    Le scelte si possono sperimentare anticipatamente & Comunicazione più complicata tra le persone coinvolte \\
    \hline
\end{tabularx}

\subsection{ISO 13407: «Human-centered design process for interactive systems»}

\begin{center}
    \begin{tikzpicture}[
            % Definizione dello stile per i blocchi grigi
            block/.style={
                    rectangle,
                    draw=black,
                    fill=gray!20,
                    text width=4cm,
                    align=center,
                    minimum height=1.5cm,
                    font=\sffamily\small
                },
            % Stile per le frecce
            line/.style={
                    -Stealth,
                    thick,
                    draw=black
                }
        ]

        % 1. Posizionamento dei nodi
        % Nodo iniziale in alto
        \node [block] (start) {Identificare le necessità per la progettazione human-centered};

        % Ciclo centrale
        \node [block, below=1cm of start] (context) {Comprendere e specificare il contesto d'uso};
        \node [block, right=1.5cm of context, yshift=-2.5cm] (req) {Specificare i requisiti utente e organizzativi};
        \node [block, below=2.5cm of context, yshift=-2.5cm] (sol) {Produrre soluzioni di progetto};
        \node [block, left=1.5cm of context, yshift=-2.5cm] (eval) {Valutare il progetto rispetto ai requisiti};

        % 2. Disegno delle frecce
        % Freccia dritta dall'inizio
        \draw [line] (start) -- (context);

        % Frecce curve del ciclo (usiamo bend per l'effetto circolare)
        \draw [line] (context.east) to[bend left=30] (req.north);
        \draw [line] (req.south) to[bend left=30] (sol.east);
        \draw [line] (sol.west) to[bend left=30] (eval.south);
        \draw [line] (eval.north) to[bend left=30] (context.west);

        % 3. Freccia di uscita (Il sistema soddisfa...)
        \draw [line] (eval.north) -- ++(0, 2cm)
        node[above, text width=3cm, align=center, font=\sffamily\small]
        {il sistema soddisfa i requisiti utente e organizzativi};

    \end{tikzpicture}
\end{center}

\section*{Processo di Progettazione Human-Centered (ISO 13407)}

Il processo definito dalla norma ISO 13407 prevede quattro fasi cicliche principali, precedute dalla pianificazione iniziale

\begin{enumerate}
    \item \bld{Comprendere e specificare il contesto d'uso}
          \begin{itemize}
              \item Analisi delle caratteristiche degli utenti, incluse competenze, abilità, esperienze e preferenze
              \item Descrizione dei compiti che gli utenti dovranno eseguire, considerando frequenza, durata e implicazioni per la sicurezza.
              \item Definizione dell'ambiente fisico e sociale, comprese le tecnologie utilizzate e gli standard organizzativi.
          \end{itemize}

    \item \bld{Specificare i requisiti degli utenti e dell'organizzazione}
          \begin{itemize}
              \item Definizione delle prestazioni richieste rispetto agli obiettivi operativi ed economici.
              \item Identificazione dei requisiti normativi, legislativi e di sicurezza.
              \item Analisi della comunicazione e cooperazione tra utenti e altri attori coinvolti.
              \item Progettazione delle attività e delle procedure di lavoro, inclusa la gestione del cambiamento e dell'addestramento.
              \item Valutazione della fattibilità delle operazioni e della progettazione dei posti di lavoro e delle interfacce.
          \end{itemize}

    \item \bld{Produrre soluzioni di progetto}
          \begin{itemize}
              \item Sviluppo di proposte progettuali attraverso un approccio multidisciplinare.
              \item Realizzazione di simulazioni, modelli e prototipi per rendere concrete le soluzioni.
              \item Presentazione dei progetti agli utenti per l'esecuzione di compiti simulati.
              \item Modifica iterativa del progetto in base alle reazioni degli utenti fino al raggiungimento degli obiettivi.
          \end{itemize}

    \item \bld{Valutare il progetto rispetto ai requisiti}
          \begin{itemize}
              \item Definizione di un piano di valutazione e produzione di feedback per il team di progettazione.
              \item Verifica del raggiungimento degli obiettivi prefissati e validazione sul campo.
              \item Monitoraggio a lungo termine del sistema e documentazione sistematica dei risultati ottenuti.
          \end{itemize}
\end{enumerate}

Il ciclo si conclude quando il sistema soddisfa pienamente i requisiti utente e organizzativi identificati.
