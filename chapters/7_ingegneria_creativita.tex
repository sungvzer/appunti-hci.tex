\section{Ingegneria e creatività}

La progettazione non è strettamente algoritmica, ma prevede una certa dose di \bld{creatività}. Nel passaggio $\text{requisiti}\to\text{design concept}\to\text{idea}\to\text{implementazione}$ il progettista deve:

\begin{itemize}
    \item inventare nuove soluzioni ai problemi posti
    \item rappresentarle con notazione rigorosa
    \item valutarne la validità
\end{itemize}

\subsection{Tecniche di progettazione}

\subsubsection{Mimesi}

Riproduzione di un sistema esistente con tecnologie differenti, ad esempio microfoni digitali, applicazioni per bussola, calcolatrici digitali.

Questa tecnica funziona bene quando l'oggetto reale si presta bene a una sua rappresentazione virtuale, e ne può beneficiare anche acquisendo funzionalità solo realizzabili tramite software (\itc{esempio: un righello che cambia scala cliccandoci sopra}).

\subsubsection{Ibridazione}

Si concepisce un nuovo oggetto o funzionalità mischiando e integrando aspetti di oggetti e funzionalità diverse. (\itc{esempio: un software per il missaggio (console) e il download di tracce audio (iTunes)}).

\subsubsection{Metafora}

Si mescolano tra loro campi semantici differenti, trasferendo proprietà e concetti da un \bld{donatore} a un \bld{ricevente} (\itc{esempio: un menù con funzione composto solo da strisce colorate (ricevente), acquisisce significato con un'icona di un evidenziatore (donatore)}).

\subsubsection{Variazione}

Versione evoluta e migliorata di un prodotto esistente. Partendo da un prodotto, e accumulando modifiche anche di lieve entità, si può arrivare a un prodotto completamente diverso da quello di partenza (nave di Teseo).

I software sono manufatti evolutivi, le migliorie suggerite, le nuove specifiche, e gli errori che si verificano nella programmazione, fanno sì che i software vengano \bld{continuamente} modificati; il software diventa così un'entità fluida, che si trasforma in modo continuo durante il suo arco di vita.
