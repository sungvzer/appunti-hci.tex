\section{Prototipi}

Per rappresentare dei concetti durante la progettazione si usano diversi strumenti, formali e informali. All'interno dei diagrammi informali troviamo: schizzi, storyboard, ecc.

\subsection{Statechart}

Per rappresentare in modo più adeguato l'interazione con l'utente si usano i diagrammi per macchine a stati, costituiti da nodi e archi: ogni nodo è uno stato, ogni arco è una transizione da uno stato all'altro.

\begin{wrapfigure}{r}{0.25\textwidth}
    \centering
    \begin{tikzpicture}[
        state/.style={
                rectangle,
                draw=black,
                thick,
                minimum width=1.8cm,
                minimum height=1cm,
                rounded corners=8pt,
                font=\bfseries
            },
        arrow/.style={-{Stealth}, thick}
        ]
        % Nodi
        \node[state] (S1) {$S_1$};
        \node[state] (S2) [below=1.2cm of S1] {$S_2$};

        % Transizione
        \draw[arrow] (S1) -- (S2)
        node[midway, right] {\small E [C] / A};
    \end{tikzpicture}
\end{wrapfigure}

Quando il sistema si trova nello stato $S_1$, se si verifica l'evento $E$, con condizione $C$, il sistema effettua l'azione $A$ e si sposta nello stato $S_2$.

Di solito gli eventi sono associati ad azioni dell'utente.

Gli statechart hanno il vantaggio di esplicitare tutti i percorsi che l'utente può seguire durante l'interazione col sistema, evidenziando possibili punti deboli e aspetti critici del caso d'uso ancor prima di realizzare un prototipo.

\subsection{Prototipo}

Secondo lo standard ISO 13407, un prototipo è una «rappresentazione di un prodotto o sistema, o di una sua parte che, anche se limitata, può essere usata a scopo di valutazione». Un prototipo è quindi un \bld{modello approssimativo} del sistema, che serve solo a scopo valutativo.

Usare un prototipo permette di:

\begin{itemize}
    \item rendere le decisioni di progetto esplicite
    \item esplorare più design e varianti del progetto
    \item incorporare il feedback degli utenti
    \item migliorare la qualità delle specifiche
\end{itemize}

Un prototipo se usato bene è un valido aiuto in quanto può essere sviluppato velocemente, a basso costo, e diventa uno strumento di comunicazione per il gruppo di lavoro; bisogna però identificare le parti da tenere fisse, e quelle da modificare.

I prototipi usano strumenti poco efficienti, dovuti anche a codice di bassa qualità e algoritmi semplificati.

\subsection{Classificazione dei prototipi}

\subsubsection{Scopo del prototipo}

Ogni prototipo ha un suo scopo:

\noindent % Impedisce l'indentazione del blocco
\begin{minipage}[t]{0.65\textwidth} % Blocco testo (65%)
    \vspace{0pt} % Allinea in alto con la figura
    \begin{itemize}
        \item \textbf{ruolo}: si sperimenta il ruolo del prodotto nella vita degli utenti
        \item \textbf{user experience}: si sperimenta l'esperienza d'uso
        \item \textbf{implementazione}: si sperimentano tecniche e componenti per la realizzazione finale
    \end{itemize}
\end{minipage}
\hfill % Spazio elastico tra i due blocchi
\begin{minipage}[t]{0.3\textwidth} % Blocco figura (30%)
    \vspace{0pt} % Allinea in alto con il testo
    \centering
    \begin{tikzpicture}[scale=0.6, every node/.style={scale=0.7}]
        % Definizione dei punti del triangolo
        \coordinate (Top) at (90:2.5);
        \coordinate (Left) at (210:2.5);
        \coordinate (Right) at (330:2.5);

        % Disegno del triangolo
        \draw[thick] (Top) -- (Left) -- (Right) -- cycle;

        % Etichette ai vertici
        \node[above=0.1cm of Top] {\textbf{Ruolo}};
        \node[below left=0.1cm and -0.7cm of Left] {Implementazione};
        \node[below right=0.1cm and -0.7cm of Right] {Interfaccia};

        % Il puntino verde sottile
        \filldraw[fill=green!60!black, draw=black, thin] (-0.8,-0.5) circle (3pt);
    \end{tikzpicture}
\end{minipage}
\vspace{1em} % Spazio dopo il blocco

\subsubsection{Prototipi statici, dinamici, interattivi}

La rappresentazione \bld{statica} del prodotto è una serie d'immagini, un modello 3D, o una rappresentazione statica del funzionamento dinamico del prodotto. I prototipi \bld{dinamici} invece sono una sequenza d'immagini, video, ecc. che vengono mostrati all'utente in sequenza, simulando scenari d'uso tipici. I prototipi più utili sono \bld{interattivi}, poiché sono utilizzabili dagli utenti durante i test.

\subsubsection{Fedeltà dei prototipi}

Un prototipo a \bld{bassa fedeltà} assomiglia in modo approssimativo al prodotto finale: parliamo di \itc{mockup}. Un prototipo ad \bld{alta fedeltà} invece è quanto più possibile vicino agli aspetti del prodotto finale.

\subsubsection{Ciclo di vita del prototipo}

Il primo tipo di ciclo di vita è il prototipo \itc{throw away}: prototipi usati poche volte e ignorati.

IL prototipo evolutivo invece parte da un prototipo iniziale ed evoluto fino alla versione finale: tende a fissare una soluzione sin dall'inizio del progetto.

I prototipi incrementali sono sviluppati e installati a fasi, mentre quelli interattivi sono modificati al volo in base ai commenti dell'utente.

\subsubsection{Prototipi orizzontali e verticali}

I prototipi vengono classificati per la loro completezza funzionale, un prototipo \bld{orizzontale} è un prototipo che non permette di svolgere alcun task fino in fondo, ma permette di valutare l'intera interfaccia, mentre un prototipo \bld{verticale} permette di testare solo una parte di sistema fino in fondo, in situazioni realistiche, con utenti reali.
