\section{Introduzione al corso}

\subsection{Info sul corso}

\begin{itemize}
    \item CdS 0124 Informatica
    \item CFU: 3
    \item Docente: Prof.ssa Mariacarla Staffa
    \item Email: \href{mailto://mariacarla.staffa@uniparthenope.it}{mariacarla.staffa@uniparthenope.it}
    \item Ricevimento: Mercoledì 14:30 - 15:30 (su teams o in presenza previa email) codice del team: 6ufbo3f

\end{itemize}

\subsection{Obiettivi del corso}

Si vuole fornire un'introduzione pratica e teorica alle problematiche del design dell'interazione uomo macchina, per la progettazione di sistemi interattivi facili da usare. Si studieranno i modelli, gli stili, e i paradigmi dell'interazione.
\textit{Usabile} significa facile da apprendere, pratico da usare, e il cui uso fornisce un'esperienza piacevole.

\subsection{Libri di testo}
J. Preece, Y. Rogers, H. Sharp. \itc{Interaction Design}. Wiley.\\
A. Dix, J. Finlay, G. Abowd, R. Beale. \itc{Interazione Uomo Macchina}. McGraw-Hill.\\
B. Shneiderman, C. Plaisant. \itc{Designing the User Interface -- Strategies for effective Human-Computer interaction}. 5° Edizione. Addison-Wesley.

\subsection{Verifica dell'apprendimento}

Ogni studente parteciperà alla realizzazione di un progetto di gruppo (3-5 studenti), su cui verranno accertate le capacità acquisite. La verifica si baserà su:

\begin{itemize}
    \item La compilazione di un report che mira a verificare che lo studente abbia acquisito la capacità d'ideare, progettare, sviluppare e valutare prototipi d'interfacce utente usabili, anche grazie a un’opportuna modellazione degli utenti in un determinato contesto d’uso.
    \item Una prova orale dove gli studenti presenteranno (anche attraverso delle slides o dei prototipi realizzati durante il corso in Android o XML) il progetto realizzato.
\end{itemize}

\section{L'interazione uomo macchina}
L'interazione uomo macchina si occupa della progettazione, valutazione e implementazione di sistemi di calcolo interattivo per uso umano e dello studio dei principali fenomeni che li circondano. L'obiettivo è quello di progettare sistemi interattivi facili da usare, efficienti, efficaci e che forniscano un'esperienza piacevole agli utenti.

\subsection{Le interfacce utente mobile}
In ambito mobile, gli aspetti che si possono ``toccare'' sono molti:

\begin{itemize}
    \item L'approccio tecnico
    \item La progettazione, concentrata sull'interazione tra utente e applicazione
\end{itemize}

Quando si progetta un'interfaccia utente, ci si deve preoccupare di molto di più del semplice design, guardando aspetti come le \bld{necessità dell'utente tipico}
oppure il \bld{contesto} dell'interfaccia utente all'interno del sistema.

Ogni interfaccia viene visualizzata su display caratterizzati da \bld{densità} di pixel (PPI), \bld{spazio colore}, \bld{risoluzione}, e \bld{grandezza} del display.
Poiché la gamma di display possibili è molto vasta, l'interfaccia deve essere unica e sapersi adattare al dispositivo su cui viene eseguita.

Nell'ambito mobile è importante la coerenza col resto del sistema, tramite l'utilizzo di temi, pratiche comuni, e pattern già conosciuti agli utenti (es. Material Design).

Quando si usa un touch screen, si possono abbinare sequenze di movimenti delle dita sul display a determinati comandi. L'utente usa l'app una collezione di \itc{gestures} che ha già
interiorizzato, e rappresentano un linguaggio universale per l'interazione (\textit{touch}, \itc{swipe}, \itc{pinch}).

Il progettista di un'interfaccia deve saper quindi bilanciare quella che è la \bld{familiarità} dell'interfaccia e l'\textbf{originalità} della stessa.

L'esperienza d'interazione dell'utente con l'interfaccia viene misurata tramite l'\textbf{usabilità}. In questo caso l'interfaccia funge da filtro della complessità del sistema
sottostante, presentando all'utente un'immagine semplificata del prodotto.

Quando si prende in considerazione quindi l'interazione uomo-macchina, la progettazione diventa \bld{incentrata sull'essere umano}.

\subsection{Cos'è l'interazione uomo macchina?}

È una disciplina che si occupa di progettare, validare, e implementare sistemi informatici interattivi per uso umano e allo studio dei principali fenomeni che li circondano. L'HCI nasce da aree disciplinari molto diverse:

\begin{itemize}
    \item Ergonomia
    \item Scienza dei computer
    \item Psicologia
    \item Sociologia e antropologia
\end{itemize}
