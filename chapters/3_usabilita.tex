\section{Usabilità}

\subsection{Ciclo di feedback}

È il modello più semplice dell'interazione sistema-utilizzatore: l'utente fornisce un input al sistema, e il sistema risponde (\itc{feedback}).

\subsection{Modello di Donald Norman}

È diviso in sette fasi:

\begin{enumerate}
    \item \bld{Formazione di uno scopo}
    \item \bld{Formazione di un'intenzione}: cosa vogliamo fare per raggiungere lo scopo
    \item \bld{Specifica di un'azione}: si pianificano nel dettaglio le azioni da compiere
    \item \bld{Esecuzione}: si eseguono le azioni pianificate
    \item \bld{Percezione dello stato del mondo}: si osserva come cambia il sistema e il mondo circostante dopo le azioni
    \item \bld{Interpretazione dello stato del mondo}: si elabora ciò che si percepisce
    \item \bld{Valutazione del risultato}: si decide se lo scopo è stato raggiunto
\end{enumerate}

\subsubsection{Esempio: aprire l'acqua per fare la doccia}

Ipotizziamo che io voglia fare una doccia:

\begin{enumerate}
    \item \bld{Formazione di uno scopo}: voglio aprire l'acqua per fare la doccia
    \item \bld{Formazione di un'intenzione}: voglio operare sul rubinetto
    \item \bld{Specifica di un'azione}: ... alzando la maniglia verso l'alto e verso sinistra, fino in fondo
    \item \bld{Esecuzione}: apro il rubinetto secondo la specifica
    \item \bld{Percezione dello stato del mondo}: vedo l'acqua uscire dalla doccia, sento l'acqua calda, il rubinetto non si muove più di così
    \item \bld{Interpretazione dello stato del mondo}: il rubinetto è aperto, il flusso dell'acqua calda è conseguenza delle mie azioni
    \item \bld{Valutazione del risultato}: decido di aver raggiunto il mio scopo
\end{enumerate}

\subsubsection{Golfo dell'esecuzione}

Quantifica il grado in cui il sistema risponde alle azioni dell'utente, ovvero se le azioni permesse dal sistema coincidono con le intenzioni dell'utente.

L'obiettivo è quello di \bld{minimizzare} questo divario.

\subsubsection{Golfo della valutazione}

Quantifica lo sforzo richiesto a una persona per interpretare lo stato del sistema, e quanto aspettative e intenzioni sono state soddisfatte.

L'obiettivo è quello di \bld{minimizzare} la differenza tra le aspettative e i risultati effettivi.

\subsubsection{Affordance \& Feedback}

L'\bld{affordance} è la proprietà di un oggetto d'influenzare visivamente il modo in cui viene usato: riduce il golfo dell'esecuzione.

Il \bld{feedback}, invece, dev'essere facilmente interpretabile e specifico, così da permettere all'utente di sapere che risultati hanno avuto le sue azioni: riduce il golfo della valutazione.

\subsection{Definizione di Usabilità}

È l'efficacia, efficienza, e la soddisfazione con cui gli utenti raggiungono determinati obiettivi in determinati ambienti d'uso.

Definiamo quindi:

\begin{itemize}
    \item Efficacia: accuratezza e completezza con cui raggiungo un obiettivo; considera il livello di precisione con cui l'utente arriva ai suoi scopi
    \item Efficienza: risorse usate per ottenere il risultato
    \item Soddisfazione: comfort e accettabilità del sistema
\end{itemize}

Un aspetto importante della misurazione dell'usabilità è \bld{quando} questa viene misurata (\itc{Al primo utilizzo? Quando un utente è diventato esperto?})

\subsubsection{Learnability}

Quanto è facile imparare a usare un sistema. Un progettista può scegliere se progettare un sistema per utenti occasionali, continuativi, o entrambi, applicando una determinata curva di apprendimento alla sua interfaccia.

\subsubsection{Memorability}

Quanto è facile ricordare come si usa un sistema per usi occasionali. Più un sistema è memorabile, quanto più è veloce il riapprendimento di azioni e nozioni.

\subsection{Sussidi dell'utente}

Un sistema interattivo è corredato da una serie di sussidi che permettono all'utente di usarlo agevolmente, ad esempio help desk, manuali utente, assistenza clienti, ecc. Spesso questi sussidi possono essere integrati stesso nel prodotto.

Un sistema usabile dovrebbe mettere in grado i suoi utenti di usarlo senza alcun tipo di sussidio esterno.

\subsection{Usabilità generale}

Prodotti e servizi destinati a utenze generiche e usabili da tutti si definiscono \bld{usabili universalmente}.

\subsection{Usabilità e Accessibilità}

Per accessibilità si intende la capacità dei sistemi informatici di erogare servizi e fornire informazioni senza discriminazioni per disabilità, o utenze disagiate.

Un sistema usabile non implica che sia pure accessibile, e viceversa.
