\section{Progettare per l'utente}

C'è una differenza sostanziale tra \itc{progettare} e \itc{realizzare}, quando si progetta si immagina qualcosa e si studia come attuarla, mentre la realizzazione è rendere quel qualcosa reale attuandolo praticamente.

Nella progettazione tradizionale si pensava al sistema come oggetto principale da progettare, oggi invece si attua una progettazione centrata sull'essere umano: \itc{human-centred design}, o HCD.

\subsection{Requisiti di prodotto}

Sono le priorità che si richiedono al prodotto, raccolte per iscritto in un documento strutturato, e forniranno poi l'input alle attività di progettazione.

L'approccio moderno per la raccolta dei requisiti è il seguente: ci si chiede quali sono i \bld{casi d'uso} dell'utente rispetto al sistema, e in seguito si progetta l'interazione necessaria.

\subsection{Casi d'uso}

Un caso d'uso è definibile come un'«insieme d'interazioni fra uno o più utenti e il sistema, finalizzate a uno scopo utile per l'utente».

È sconnesso dal concetto di funzionalità di sistema: un caso d'uso può essere implementato mediante più funzioni di sistema.
