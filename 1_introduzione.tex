\section{Introduzione}

\subsection{Info sul corso}

\begin{itemize}
	\item CdS 0124 Informatica
	\item CFU: 3
	\item Docente: Prof.ssa Mariacarla Staffa
	\item Email: \href{mailto://mariacarla.staffa@uniparthenope.it}{mariacarla.staffa@uniparthenope.it}
	\item Ricevimento: Mercoledì 14:30 - 15:30 (su teams o in presenza previa email) codice del team: 6ufbo3f

\end{itemize}

\subsection{Obiettivi del corso}

Si vuole fornire un'introduzione pratica e teorica alle problematiche del design dell'interazione uomo macchina, per la progettazione di sistemi interattivi facili da usare. Si studieranno i modelli, gli stili, e i paradigmi dell'interazione.
\textit{Usabile} significa facile da apprendere, pratico da usare, e il cui uso fornisce un'esperienza piacevole.

\subsection{Libri di testo}
J. Preece, Y. Rogers, H. Sharp. \textit{Interaction Design}. Wiley.\\
A. Dix, J. Finlay, G. Abowd, R. Beale. \textit{Interazione Uomo Macchina}. McGraw-Hill.\\
B. Shneiderman, C. Plaisant. \textit{Designing the User Interface -- Strategies for effective Human-Computer interaction}. 5° Edizione. Addison-Wesley.

\subsection{Verifica dell'apprendimento}

Ogni studente parteciperà alla realizzazione di un progetto di gruppo (3-5 studenti), su cui verranno accertate le capacità acquisite. La verifica si baserà su:

\begin{itemize}
	\item La compilazione di un report che mira a verificare che lo studente abbia acquisito la capacità di ideare, progettare, sviluppare e valutare prototipi di interfacce utente usabili, anche grazie a un’opportuna modellazione degli utenti in un determinato contesto d’uso.
	\item Una prova orale dove gli studenti presenteranno (anche attraverso delle slides o dei prototipi realizzati durante il corso in android o xml) il progetto realizzato.
	\item Test123
\end{itemize}
