\section{Conoscere l'utente}

L'usabilità è una proprietà relativa allo specifico utente, al compito e al contesto di utilizzo: un sistema usabile per un utente potrebbe essere inutilizzabile da un altro. Conoscere l'utente è quindi fondamentale per chi progetta un sistema.

\subsection{User-centered design}

L'utente va al centro delle attività del design: è il sistema che si adatta all'utente, non il contrario.

\subsection{Livelli di descrizione dell'utente}

L'utente si può descrivere su vari piani, in funzione degli aspetti esaminati.

Gli utenti possono rivestire uno o più ruoli diversi in base alla funzione che esercitano nel sistema.

\itc{N.B. Persona e ruolo sono due concetti distinti! Due utenti potrebbero ricoprire lo stesso ruolo, o un utente potrebbe ricoprire più ruoli.}

\section{Human Information Processor}

È un modello di modellazione cognitiva che si divide in tre sottosistemi interagenti: percettivo, motorio e cognitivo.

\subsection{Attenzione}

In questi tre sottosistemi ricopre un ruolo fondamentale l'\bld{attenzione}, ovvero quei processi cognitivi che ci permettono di selezionare le informazioni più utili tra tutte quelle che arrivano al cervello contemporaneamnte. Si divide in:

\begin{itemize}
    \item \bld{attenzione selettiva}: quando ci focalizziamo su un singolo evento escludendo gli altri
    \item \bld{attenzione divisa}: quando seguiamo contemporaneamente più eventi; ha possibilità limitate, non riusciamo a prestare attenzione a troppe cose contemporaneamente
\end{itemize}

\begin{center}
    \resizebox{\textwidth}{!}{
        \begin{tikzpicture}[
            font=\sffamily\small,
            % Processori (Ellissi scure)
            proc/.style={ellipse, draw, fill=gray!60, text=white, align=center, minimum width=2.5cm, minimum height=1.5cm},
            % Memorie (Rettangoli chiari)
            mem/.style={rectangle, draw, fill=gray!20, align=center, minimum width=2.2cm, minimum height=1.2cm},
            % Attenzione
            attn/.style={ellipse, draw, minimum width=4.5cm, minimum height=1cm},
            % Stili Frecce
            singleArrow/.style={-{Stealth[scale=1.2]}, thick, draw=olive!70!black},
            doubleArrow/.style={{Stealth[scale=1.2]}-{Stealth[scale=1.2]}, thick, draw=olive!70!black},
            line/.style={thick, draw=olive!70!black}
            ]

            % --- 1. POSIZIONAMENTO NODI ---
            \node (sense) {Organi di senso};
            \node [mem, right=0.8cm of sense] (sensMem) {Memoria\\sensoriale};
            \node [proc, right=0.8cm of sensMem] (percProc) {Processore\\percettivo};
            \node [proc, right=1.2cm of percProc] (cognProc) {Processore\\cognitivo};
            \node [proc, right=1.2cm of cognProc] (motProc) {Processore\\motorio};
            \node [below=0.6cm of motProc] (muscles) {Muscoli};

            % Memorie inferiori (posizionate rispetto ai processori)
            \node [mem, below=1.2cm of $(percProc.south)!0.5!(cognProc.south)$] (stMem) {Memoria a\\breve termine};
            \node [mem, below=1.2cm of $(cognProc.south)!0.5!(motProc.south)$] (ltMem) {Memoria a\\lungo termine};

            % Attenzione (in alto)
            \node [attn, above=1.5cm of cognProc] (attn) {Attenzione};

            % --- 2. COLLEGAMENTI ---
            % Percorso orizzontale
            \draw [singleArrow] (sense) -- (sensMem);
            \draw [singleArrow] (sensMem) -- (percProc);
            \draw [singleArrow] (percProc) -- (cognProc);
            \draw [singleArrow] (cognProc) -- (motProc);
            \draw [line, dashed] (motProc) -- (muscles);

            % Collegamenti Memorie (DOPPIE FRECCE)
            \draw [singleArrow] (stMem.north) -- (percProc.south);
            \draw [doubleArrow] (stMem.east) -- (ltMem.west);
            \draw [doubleArrow] (cognProc.south) -- (ltMem.north);

            % Collegamenti Attenzione (Frecce grigie)
            \draw [singleArrow, gray] (attn.210) -- (percProc.north);
            \draw [singleArrow, gray] (attn.270) -- (cognProc.north);
            \draw [singleArrow, gray] (attn.330) -- (motProc.north);
            % Freccia curva verso memoria a lungo termine
            \draw [singleArrow, gray] (attn.east) to[out=0, in=0, looseness=1.5] (ltMem.east);

            % Feedback tratteggiato
            \draw [singleArrow, dashed, gray] (muscles.west) -| (sense.south);
            \node [below, font=\sffamily\footnotesize] at ($(sense.south)!0.5!(muscles.south)$) {Feedback};

        \end{tikzpicture}
    }
\end{center}

Questo modello ha implicazioni importanti sul design dell'interazione:

\begin{itemize}
    \item Dove e come dirigere l'attenzione?
          \subitem Realizzando opportuni \itc{attention cue} che guidino l'attenzione dell'utente
    \item Come mantenere l'attenzione?
    \item Come evito interferenze?
\end{itemize}

\subsection{Memoria}

Questo modello, chiaramente ispirato all’information processing, ipotizza l’esistenza di una serie di fasi, attraverso cui l’informazione transita, e una serie di “magazzini” destinati a contenerla.


\begin{center}
    \resizebox{\linewidth}{!}{%
        \begin{tikzpicture}[
            font=\sffamily\small,
            % Ogni nodo che usa \\ deve avere align=center
            block/.style={
                    rectangle, draw, fill=gray!20, rounded corners,
                    minimum width=3.5cm, minimum height=1.5cm, align=center
                },
            % Stile per i testi sulle frecce e nodi semplici
            labelnode/.style={align=center},
            arrow/.style={-{Stealth}, line width=1.5pt}
            ]

            % 1. Nodi principali
            \node[labelnode] (input) {input\\sensoriale};
            \node[block, right=1.5cm of input] (sens) {Memoria\\sensoriale};
            \node[block, right=2.5cm of sens] (short) {Memoria a\\breve\\termine};
            \node[block, right=2.5cm of short] (long) {Memoria a\\lungo\\termine};

            % 2. Frecce orizzontali
            \draw[arrow] (input) -- (sens);
            \draw[arrow] (sens) -- node[above, labelnode] {attenzione} (short);

            % 3. Codifica e Recupero (spostati con yshift per non sovrapporsi)
            \draw[arrow] ([yshift=0.3cm]short.east) -- node[above, labelnode] {codifica} ([yshift=0.3cm]long.west);
            \draw[arrow] ([yshift=-0.3cm]long.west) -- node[below, labelnode] {recupero} ([yshift=-0.3cm]short.east);

            % 4. Loop di ripetizione (risolto errore semicolon e LR mode)
            \draw[arrow] (short.north) .. controls +(up:1.2cm) and +(left:1.2cm) ..
            node[above, labelnode] {ripetizione\\di mantenimento} (short.120);

            % 5. Frecce di perdita (oblio) verso il basso
            \draw[arrow] (sens.south) -- ++(0,-1.2)
            node[below, labelnode, text width=3.5cm] {l'informazione non selezionata è perduta rapidamente};

            \draw[arrow] (short.south) -- ++(0,-1.2)
            node[below, labelnode, text width=3.5cm] {le informazioni non ripetute sono perdute rapidamente};

            \draw[arrow] (long.south) -- ++(0,-1.2)
            node[below, labelnode, text width=3.5cm] {alcune informazioni sono perdute col tempo};

        \end{tikzpicture}%
    }
\end{center}

\subsubsection{Memoria a breve termine}

Il massimo numero di raggruppamenti di elementi che la mente riesce a trattenere nel breve termine è tra i 5 e i 9, e la permanenza in questa memoria oscilla tra i 15 e i 30 secondi.

La persistenza aumenta con la \bld{ripetizione} e l'attenzione, spostando un'informazione dalla memoria a breve termine in quella a lungo termine.

Un bravo progettista non sovraccarica la STM di un utente, o cerca di minimizzarne in ogni caso l'utilizzo.

Tra un task e un altro tendiamo a scaricare informazioni dalla memoria a breve termine, motivo per cui ogni task dev'essere ben definito e semplice, e i task devono essere definiti in modo sequenziale.

\subsubsection{Memoria a breve termine}

Le interferenze (segnali acustici, visivi, ecc.) possono creare interruzioni nei processi cognitivi, generando rallentamenti e stress. Ad esempio, l'effetto Stroop:

\begin{center}
    Pronuncia i \bld{colori} di queste scritte:

    \textcolor{red}{Casa}\\
    \textcolor{blue}{Rosso}\\
    \textcolor{orange}{Bambino}\\
    \textcolor{green}{Treno}
\end{center}

\subsection{Memoria a lungo termine}

Per fini pratici  è illimitata, ha un tempo di accesso lungo ed ha una persistenza di lunghissima durata.

Si introducono i concetti di \bld{rievocazione} (prelevare un'informazione dalla memoria) e \bld{riconoscimento} (confrontare un'informazione data con il contenuto memorizzato). Per un utente è più facile riconoscere che rievocare (ad esempio, è più facile ricordare la funzione del pulsante salva con il floppy disk piuttosto che ricordare il comando specifico per chiudere il computer da terminale).

Nel design quindi si preferisce il riconoscimento alla rievocazione, con associazioni visive forti, ripetute, e distinte.
